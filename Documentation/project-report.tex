\documentclass[conference]{IEEEtran}
%\IEEEoverridecommandlockouts
% The preceding line is only needed to identify funding in the first footnote. If that is unneeded, please comment it out.
\usepackage{cite}
\usepackage{amsmath,amssymb,amsfonts}
\usepackage{algorithmic}
\usepackage{graphicx}
\usepackage{textcomp}
\usepackage{xcolor}
\def\BibTeX{{\rm B\kern-.05em{\sc i\kern-.025em b}\kern-.08em
    T\kern-.1667em\lower.7ex\hbox{E}\kern-.125emX}}
\begin{document}

\title{Evolutionary Music%\\
%{\footnotesize \textsuperscript{*}Note: Sub-titles are not captured in Xplore and
%should not be used}
%\thanks{Identify applicable funding agency here. If none, delete this.}
}

\author{\IEEEauthorblockN{Nisha Shareef Shaikh}
\IEEEauthorblockA{\textit{dept. name of organization (of Aff.)} \\
\textit{name of organization (of Aff.)}\\
Karachi, Pakistan \\
ns02530@st.habib.edu.pk}
\and
\IEEEauthorblockN{Abeera Tariq}
\IEEEauthorblockA{\textit{dept. name of organization (of Aff.)} \\
\textit{name of organization (of Aff.)}\\
Karachi, Pakistan \\
at02787@st.habib.edu.pk}
}

\maketitle

\begin{abstract}
*CRITICAL: Do Not Use Symbols, Special Characters, Footnotes, 
or Math in Paper Title or Abstract.
\end{abstract}

\begin{IEEEkeywords}
music, evolution, evolutionary algorithm
\end{IEEEkeywords}

\section{Introduction}
Motivation, Novelty, How your paper is tackling it and what it does well\\
\\
Music composition involves innovation, creativity and sense of melody. Through this project, we hope to discover how music can be composed without the involvement of musicians. Taking motivation from the paper we presented for our first presentation, we will try to utilize our knowledge of evolutionary algorithms to compose melodies.


\section{Technical Background}
Explain techniques used in the project\\\\
Evolutionary Algorithms as inspired by the process of biological evolution work towards optimization. Using concepts of reproduction, mutation, selection and recombination on candidate solutions (chromosomes), evolutionary algorithms (also called genetic algorithms) solve an optimization problem. Fitness function determines how good the candidate solution is and works by evolving the solutions to obtain the optimum solution to the problem.


\section{Related Work}
How other people have done it, What problems they faced\\\\

\section{Methodology}
Our approach

\section{Experiment and Results}
How we tested\\
What benchmarks we used to evaluate results\\
Dataset used\\
Experiment settings\\
Results and comparison if done\\
Analysis of results (whether they are good; if they are bad, then in what situations are they bad)

\section{Conclusion}
Including future work

\section*{Acknowledgment}
What resources we have used and who has helped us

\section*{References}

Please number citations consecutively within brackets \cite{b1}. The 
sentence punctuation follows the bracket \cite{b2}. Refer simply to the reference 
number, as in \cite{b3}---do not use ``Ref. \cite{b3}'' or ``reference \cite{b3}'' except at 
the beginning of a sentence: ``Reference \cite{b3} was the first $\ldots$''

Number footnotes separately in superscripts. Place the actual footnote at 
the bottom of the column in which it was cited. Do not put footnotes in the 
abstract or reference list. Use letters for table footnotes.

Unless there are six authors or more give all authors' names; do not use 
``et al.''. Papers that have not been published, even if they have been 
submitted for publication, should be cited as ``unpublished'' . Papers 
that have been accepted for publication should be cited as ``in press'' . 
Capitalize only the first word in a paper title, except for proper nouns and 
element symbols.

For papers published in translation journals, please give the English 
citation first, followed by the original foreign-language citation .

\begin{thebibliography}{00}
\bibitem{b1} D. Matic, "A genetic algorithm for composing music", \textit{Yugoslav Journal of Operations Research}, vol. 20, no. 1, pp. 157-177, 2010. Available: 10.2298/yjor1001157m [Accessed 19 April 2019].
\bibitem{b2} Z. Geem and J. Choi, "Music Composition Using Harmony Search Algorithm", \textit{Lecture Notes in Computer Science}, pp. 593-600. Available: 10.1007/978-3-540-71805-5\_65 [Accessed 19 April 2019].
\bibitem{b3} A. Freitas, F. Guimarães and R. Barbosa, "Automatic Evaluation Methods in Evolutionary Music: An Example with Bossa Melodies", \textit{Lecture Notes in Computer Science}, pp. 458-467, 2012. Available: 10.1007/978-3-642-32964-7\_46 [Accessed 19 April 2019].
\end{thebibliography}

\end{document}
